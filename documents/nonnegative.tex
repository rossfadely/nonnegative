\documentclass[12pt,preprint]{aastex}

% has to be before amssymb it seems
\usepackage{color,hyperref}
\definecolor{linkcolor}{rgb}{0,0,0.5}
\hypersetup{colorlinks=true,linkcolor=linkcolor,citecolor=linkcolor,
            filecolor=linkcolor,urlcolor=linkcolor}
\newcommand{\numberparagraphs}{}
\newcommand{\nonumberparagraphs}{}

\usepackage{url}
\begin{document}

\title{PSF Regularizations: Consequences and Solutions}

\newcommand{\nyu}{2}
\newcommand{\mpia}{3}
\author{Ross~Fadely\altaffilmark{\nyu},   David~W.~Hogg\altaffilmark{\nyu,\mpia}, 
Daniel~Foreman-Mackey\altaffilmark{\nyu}, \etal}
\altaffiltext{\nyu}{Center for Cosmology and Particle Physics,
                        Department of Physics, New York University,
                        4 Washington Place, New York, NY, 10003, USA}
\altaffiltext{\mpia}{Max-Planck-Institut f\"ur Astronomie,
                        K\"onigstuhl 17, D-69117 Heidelberg, Germany}

\begin{abstract}
\end{abstract}

\section{Introduction}

\begin{itemize}

\item Regularization of PSF models has an important impact on the modeling and 
measurements of astronomical images.
\item SDSS and others allow PSF models to have negative values
\item Negative-valued pixels in PSF models are counter-intuitive from a physical 
standpoint, and are tricky to incorporate into generative models
\item Forcing non-negativity causes larger and erroneous PSF models
\item New priors/regularizations are needed to fix these issues.

\end{itemize}

\section{Likelihood and PSF Model}
\label{sec:psfmodel}

We want a pixelated model for the PSF, which will be fit to a set of stars in astronomical 
images.  The pixelated model is a much more general form of the PSF than parametric 
representations, and fit in naturally when modeling the underlying intensity field.  Given 
a set of $N_s$ PSF stars, the likelihood of the PSF model is 

\begin{eqnarray}\displaystyle
\mathcal{L} &=& \prod P(D_i | f_i, M)\\
&{\rm where}&\nonumber\\
P(D_i | F_i, M) &=& \sum_j^{N_p} \frac{1}{\sqrt(2\pi\sigma_{ij}^2)}e^{-\frac{1}{2}(p_{ij}-f_i m_j)^2/\sigma_{ij}^2} \,.
\end{eqnarray}

\noindent The data $D_i$ for the $i$th star consists of $N_p \times N_p$ pixels values, 
which are modeled as (for background subtracted patches) a single flux $f_i$ times PSF 
model pixels $m_j$ over the same pixel grid.  Each data pixel has associated variance 
$\sigma_{ij}^2$.  A best-fit un-regularized model for the PSF is one which finds the values 
of $f_i,m_j$ which maximize the likelihood.  Under the presence of noisy data, values of 
$m_j$ will have negative values.  Note the procedure of maximizing the likelihood is 
simple linear least squares fitting \citep{hoggdatarecipes}.

\section{Regularizations}
\label{sec:regularizations}

While often suitable in practice, the likelihood optimization described in Section \ref{sec:psfmodel} 
is unjustified in the sense that it does not capture basic prior knowledge about the PSF.  Two 
basic beliefs held about the PSF is that it ought to be \emph{smooth}, decreasing from a peaked 
central value monotonically in a continuous fashion, and it ought to be \emph{near-zero} valued 
at large radii.   In addition, its reasonable to expect a good PSF model to sum to one, such that 
which convolved with a delta function with amplitude set to a star's flux, one has a realistic model for 
the star in an image.



\end{document}


